% Chapter 5: Complete Example
% A full physics problem demonstrating all features

\section{Complete Example: Electromagnetic Wave}

This example demonstrates multiple features of the template in a realistic context.

\subsection{Problem Statement}

\begin{importantbox}
Calculate the electric and magnetic fields of a plane electromagnetic wave propagating in free space.
\end{importantbox}

\subsection{Theory}

\begin{definition}{Electromagnetic Wave}{emwave}
An \gls{em} wave is a solution to Maxwell's equations in vacuum that satisfies:
\[
\vnabla^2 \vect{E} = \mu_0\epsilon_0 \pdd{\vect{E}}{t}, \quad
\vnabla^2 \vect{B} = \mu_0\epsilon_0 \pdd{\vect{B}}{t}
\]
\end{definition}

\begin{theorem}{Wave Equation Solution}{wavesol}
A plane wave solution has the form:
\begin{align}
\vect{E}(\vect{r}, t) &= \vect{E}_0 \cos(\vect{k} \vdot \vect{r} - \omega t) \\
\vect{B}(\vect{r}, t) &= \vect{B}_0 \cos(\vect{k} \vdot \vect{r} - \omega t)
\end{align}
where $\omega = c|\vect{k}|$ and $c = \SI{3e8}{\meter\per\second}$.
\end{theorem}

\subsection{Key Relationships}

\subsubsection{Dispersion Relation}

The \gls{wavelength} and \gls{frequency} are related by:
\[
\lambda = \frac{c}{f} = \frac{2\pi c}{\omega}
\]

For a wave at $f = \SI{2.4}{\giga\hertz}$:
\[
\lambda = \frac{\SI{3e8}{\meter\per\second}}{\SI{2.4e9}{\hertz}} = \SI{12.5}{\centi\meter}
\]

\subsubsection{Field Relationships}

\begin{theorem}{Perpendicularity Conditions}{perp}
The electric field, magnetic field, and wave vector satisfy:
\begin{align}
\vect{E} \perp \vect{B}, \quad
\vect{E} \perp \vect{k}, \quad
\vect{B} \perp \vect{k}
\end{align}
\end{theorem}

Magnitude relationship:
\[
|\vect{B}| = \frac{|\vect{E}|}{c}
\]

\subsection{Specific Example}

\begin{example}[Plane Wave Propagating in +z Direction]
Consider a wave with $\vect{k} = k\zhat$ where $k = \SI{5.03e8}{\per\meter}$.

The electric field polarized in $\xhat$ direction:
\[
\vect{E}(z,t) = E_0 \cos(kz - \omega t) \xhat
\]

The corresponding magnetic field (from $\vect{B} = \frac{1}{c}\hat{k} \times \vect{E}$):
\[
\vect{B}(z,t) = \frac{E_0}{c} \cos(kz - \omega t) \yhat
\]

where:
\begin{itemize}
    \item $E_0 = \SI{1}{\volt\per\meter}$ (amplitude)
    \item $\omega = kc = \SI{1.51e17}{\radian\per\second}$
    \item $f = \omega/(2\pi) = \SI{2.4}{\giga\hertz}$
\end{itemize}
\end{example}

\subsection{Energy and Momentum}

\subsubsection{Energy Density}

The electromagnetic energy density:
\[
u = \frac{1}{2}\left(\epsilon_0 |\vect{E}|^2 + \frac{1}{\mu_0}|\vect{B}|^2\right)
\]

\begin{note}{Equal Contributions}{energy}
For plane waves, the electric and magnetic contributions are equal:
\[
u_E = \frac{1}{2}\epsilon_0 |\vect{E}|^2 = u_B = \frac{1}{2\mu_0}|\vect{B}|^2
\]
\end{note}

\subsubsection{Poynting Vector}

Energy flux (power per area):
\[
\vect{S} = \frac{1}{\mu_0} \vect{E} \times \vect{B}
\]

For our example:
\[
|\vect{S}| = \frac{E_0^2}{c\mu_0} \cos^2(kz - \omega t)
\]

Time-averaged intensity:
\[
\langle S \rangle = \frac{E_0^2}{2c\mu_0} = \frac{(\SI{1}{\volt\per\meter})^2}{2 \times \SI{3e8}{\meter\per\second} \times \SI{1.257e-6}{\henry\per\meter}}
\]

\subsection{Solution Summary}

\begin{solutionbox}
For a \SI{2.4}{\giga\hertz} plane wave with $E_0 = \SI{1}{\volt\per\meter}$:

\begin{enumerate}
    \item \textbf{Wavelength:} $\lambda = \SI{12.5}{\centi\meter}$
    \item \textbf{Wave vector:} $k = \SI{5.03e8}{\per\meter}$
    \item \textbf{Electric field:} $\vect{E} = E_0 \cos(kz - \omega t) \xhat$
    \item \textbf{Magnetic field:} $\vect{B} = (E_0/c) \cos(kz - \omega t) \yhat$
    \item \textbf{Power density:} $\langle S \rangle \approx \SI{1.3}{\milli\watt\per\meter\squared}$
\end{enumerate}
\end{solutionbox}

\subsection{Key Takeaways}

\begin{tipbox}{Using the Notation System}{notation-tip}
Notice how the notation system helps:
\begin{itemize}
    \item $\vect{E}, \vect{B}, \vect{k}, \vect{S}$ are clearly vectors (bold + underline)
    \item $E_0, k, \omega, c$ are clearly scalars (italic)
    \item No confusion in complex equations with many variables
\end{itemize}
\end{tipbox}

\begin{warningbox}
Always verify that $\vect{E} \perp \vect{B} \perp \vect{k}$ for plane waves. If these relationships don't hold, you may have made an error!
\end{warningbox}
