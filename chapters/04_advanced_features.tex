% Chapter 4: Advanced Features
% Demonstrates glossary, bibliography, SI units, and quantum notation

\section{Advanced Features}

\subsection{Glossary and Acronyms}

The template supports glossaries and acronyms using the \texttt{glossaries} package.

\subsubsection{Using Glossary Terms}

First mention of a term shows the full description: \gls{wavelength}.

Subsequent mentions show just the term: \gls{wavelength}.

You can also use: \Gls{frequency} (capitalized), \glspl{wavelength} (plural).

\subsubsection{Using Acronyms}

First use shows full form: \gls{em}.

Later uses show abbreviation: \gls{em}.

Force full form: \acrfull{em}.

\subsubsection{Adding Glossary Entries}

Edit \texttt{config/glossary.tex} to add your terms:
\begin{verbatim}
\newglossaryentry{myterm}{
  name=my term,
  description={Description of my term}
}

\newacronym{api}{API}{Application Programming Interface}
\end{verbatim}

\subsection{Bibliography and Citations}

The template is ready for BibTeX citations. Add entries to \texttt{references.bib}:

\begin{verbatim}
@article{shannon1948,
  author = {Shannon, Claude E.},
  title = {A Mathematical Theory of Communication},
  journal = {Bell System Technical Journal},
  year = {1948}
}
\end{verbatim}

Then cite in your document: As shown by Shannon \cite{shannon1948}, information has entropy.

To compile with bibliography:
\begin{verbatim}
pdflatex main
bibtex main
pdflatex main
pdflatex main
\end{verbatim}

\subsection{SI Units with siunitx}

The \texttt{siunitx} package provides consistent unit formatting:

\subsubsection{Basic Units}
\begin{itemize}
    \item Speed of light: \SI{3e8}{\meter\per\second}
    \item Temperature: \SI{273.15}{\kelvin}
    \item Frequency: \SI{2.4}{\giga\hertz}
    \item Energy: \SI{13.6}{\electronvolt}
\end{itemize}

\subsubsection{Ranges}
Wavelength range: \SIrange{400}{700}{\nano\meter}

Temperature range: \SIrange{0}{100}{\celsius}

\subsubsection{In Equations}
The wavelength-frequency relationship:
\[
\lambda = \frac{c}{f} = \frac{\SI{3e8}{\meter\per\second}}{\SI{2.4e9}{\hertz}} = \SI{12.5}{\centi\meter}
\]

\subsection{Quantum Mechanics Notation}

\subsubsection{Bra-Ket Notation}

Ket vector (state):
\[
\ket{\psi} = \alpha\ket{0} + \beta\ket{1}
\]

Bra vector (dual):
\[
\bra{\psi} = \alpha^*\bra{0} + \beta^*\bra{1}
\]

Inner product:
\[
\braket{\phi}{\psi} = \int_{-\infty}^{\infty} \phi^*(x)\psi(x) \dif x
\]

\subsubsection{Operators}

Hamiltonian operator:
\[
\hop{H}\ket{\psi} = E\ket{\psi}
\]

Commutator:
\[
\comm{\hop{x}}{\hop{p}} = \imag\hbar
\]

Position and momentum operators:
\[
\hop{x}\ket{x} = x\ket{x}, \quad \hop{p} = -\imag\hbar\frac{\partial}{\partial x}
\]

\subsubsection{Expectation Values}

\[
\langle \hop{A} \rangle = \braket{\psi}{\hop{A}\psi} = \int \psi^*(x) \hop{A} \psi(x) \dif x
\]

\subsection{Custom Lists with enumitem}

\subsubsection{Customized Itemize}

\begin{itemize}[label=$\triangleright$, itemsep=0pt]
    \item First item
    \item Second item
    \item Third item
\end{itemize}

\subsubsection{Customized Enumerate}

\begin{enumerate}[label=(\alph*), leftmargin=2cm]
    \item Step one
    \item Step two
    \item Step three
\end{enumerate}

\subsection{Tables with booktabs}

The template includes the \texttt{booktabs} package for professional tables:

\begin{table}[h]
\centering
\caption{Comparison of notation styles}
\begin{tabular}{lcc}
\toprule
Object & Standard & This Template \\
\midrule
Scalar & $x$ & $\scalar{x}$ \\
Vector & $\mathbf{v}$ or $\vec{v}$ & $\vec{v}$ \\
Matrix & $\mathbf{A}$ & $\mat{A}$ \\
Tensor & $\mathcal{T}$ & $\tensor{T}$ \\
\bottomrule
\end{tabular}
\end{table}
