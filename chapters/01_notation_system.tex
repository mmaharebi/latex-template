% Chapter 1: Notation System
% Demonstrates the scalar/vector/matrix/tensor notation

\section{Notation System}

This template uses a consistent visual notation system to distinguish between different mathematical objects.

\subsection{Basic Notation}

\subsubsection{Scalars}
Scalars are rendered in italic (default math mode):
\[
\scalar{x} = 5, \quad \scalar{t} = 10, \quad \scalar{\alpha} = 0.5
\]

\subsubsection{Vectors}
Vectors use bold with underline for clear distinction:
\[
\vect{v} = \begin{bmatrix} 1 \\ 2 \\ 3 \end{bmatrix}, \quad
\vect{F} = m\vect{a}
\]

Position vector example:
\[
\vect{r} = x\xhat + y\yhat + z\zhat
\]

\subsubsection{Matrices}
Matrices use bold with double underline:
\[
\matr{A} = \begin{bmatrix} 
1 & 2 & 3 \\
4 & 5 & 6 \\
7 & 8 & 9
\end{bmatrix}
\]

Matrix-vector multiplication:
\[
\matr{A}\vect{x} = \vect{b}
\]

\subsubsection{Tensors}
Higher-order tensors use bold calligraphic notation:
\[
\tensor{T} = \sum_{i,j,k} T_{ijk} \, \vect{e}_i \otimes \vect{e}_j \otimes \vect{e}_k
\]

Stress tensor in continuum mechanics:
\[
\tensor{\sigma} = \begin{bmatrix}
\sigma_{xx} & \sigma_{xy} & \sigma_{xz} \\
\sigma_{yx} & \sigma_{yy} & \sigma_{yz} \\
\sigma_{zx} & \sigma_{zy} & \sigma_{zz}
\end{bmatrix}
\]

\subsection{Unit Vectors}

The template provides unit vectors for common coordinate systems:

\paragraph{Cartesian coordinates:}
\[
\vect{v} = v_x\xhat + v_y\yhat + v_z\zhat
\]

\paragraph{Cylindrical coordinates:}
\[
\vect{E} = E_\rho\rhohat + E_\phi\phihat + E_z\zhat
\]

\paragraph{Spherical coordinates:}
\[
\vect{F} = F_R\Rhat + F_\theta\thetahat + F_\phi\phihat
\]

\subsection{Why This Notation?}

This notation system provides:
\begin{itemize}
    \item \textbf{Visual clarity} -- Each object type has distinct appearance
    \item \textbf{Reduced ambiguity} -- No confusion between $\scalar{A}$ (scalar) and $\matr{A}$ (matrix)
    \item \textbf{Better readability} -- Complex equations are easier to parse
    \item \textbf{Consistency} -- Throughout your entire document
\end{itemize}
