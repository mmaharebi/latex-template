% Chapter 3: Theorem Boxes
% Demonstrates colored theorem environments

\section{Theorem Boxes}

The template provides beautiful colored boxes for theorems, definitions, and notes.

\subsection{Numbered Theorem Environments}

\subsubsection{Theorems}

\begin{theorem}{Pythagorean Theorem}{pythag}
In a right triangle with legs of length $\scalar{a}$ and $\scalar{b}$, and hypotenuse of length $\scalar{c}$:
\[
\scalar{a}^2 + \scalar{b}^2 = \scalar{c}^2
\]
\end{theorem}

Reference it later: See Theorem~\ref{thm:pythag}.

\begin{theorem}{Fundamental Theorem of Calculus}{ftc}
If $f$ is continuous on $[a,b]$ and $F$ is an antiderivative of $f$, then:
\[
\int\limits_a^b f(x) \dif x = F(b) - F(a)
\]
\end{theorem}

\subsubsection{Lemmas and Corollaries}

\begin{lemma}{Cauchy-Schwarz Inequality}{cs}
For any vectors $\vec{u}$ and $\vec{v}$:
\[
|\vec{u} \vdot \vec{v}| \leq \norm{\vec{u}} \cdot \norm{\vec{v}}
\]
\end{lemma}

\begin{corollary}{Triangle Inequality}{triangle}
For any vectors $\vec{u}$ and $\vec{v}$:
\[
\norm{\vec{u} + \vec{v}} \leq \norm{\vec{u}} + \norm{\vec{v}}
\]
\end{corollary}

\subsubsection{Definitions}

\begin{definition}{Linear Independence}{linindep}
A set of vectors $\{\vec{v}_1, \vec{v}_2, \ldots, \vec{v}_n\}$ is \textbf{linearly independent} if:
\[
c_1\vec{v}_1 + c_2\vec{v}_2 + \cdots + c_n\vec{v}_n = \vec{0}
\]
implies $c_1 = c_2 = \cdots = c_n = 0$.
\end{definition}

\begin{definition}{Eigenvalue and Eigenvector}{eigen}
A scalar $\lambda$ is an \textbf{eigenvalue} of matrix $\mat{A}$ if there exists a non-zero vector $\vec{v}$ (the \textbf{eigenvector}) such that:
\[
\mat{A}\vec{v} = \lambda\vec{v}
\]
\end{definition}

\subsubsection{Notes}

\begin{note}{Important Observation}{note1}
The notation system helps distinguish between $\mat{A}\vec{v}$ (matrix times vector) and $\scalar{A}\vec{v}$ (scalar times vector) at a glance.
\end{note}

\subsection{Plain Theorem Environments}

For backward compatibility, plain (non-colored) environments are also available:

\begin{example}[Finding Eigenvalues]
For the matrix $\mat{A} = \begin{bmatrix} 2 & 1 \\ 1 & 2 \end{bmatrix}$, solve:
\[
\det(\mat{A} - \lambda\mat{I}) = 0
\]
\[
\det\begin{bmatrix} 2-\lambda & 1 \\ 1 & 2-\lambda \end{bmatrix} = (2-\lambda)^2 - 1 = 0
\]
Therefore $\lambda = 3$ or $\lambda = 1$.
\end{example}

\begin{remark}[On Numerical Stability]
When computing eigenvalues numerically, use specialized algorithms like QR decomposition rather than direct determinant computation.
\end{remark}

\subsection{Utility Boxes}

\subsubsection{Warning Box}

\begin{warningbox}
Division by zero is undefined! Always check denominators before computing.
\end{warningbox}

\subsubsection{Important Box}

\begin{importantbox}
Remember: matrix multiplication is NOT commutative! 
\[
\mat{A}\mat{B} \neq \mat{B}\mat{A} \quad \text{(in general)}
\]
\end{importantbox}

\subsubsection{Tip Box}

\begin{tipbox}{Computational Efficiency}{tip1}
For large sparse matrices, use specialized sparse solvers instead of dense matrix operations to save memory and computation time.
\end{tipbox}

\subsubsection{Solution Box}

\begin{solutionbox}
\textbf{Solution to Example:}
\begin{enumerate}
    \item Set up the characteristic equation
    \item Solve for eigenvalues: $\lambda_1 = 3$, $\lambda_2 = 1$
    \item Find eigenvectors by solving $(\mat{A} - \lambda\mat{I})\vec{v} = \vec{0}$
\end{enumerate}
\end{solutionbox}

\subsubsection{Answer Box}

\begin{answerbox}
Final Answer: $\lambda_1 = 3$, $\lambda_2 = 1$
\end{answerbox}
