% Chapter 6: Code Examples
% Demonstrates syntax-highlighted code boxes

\section{Code Boxes}

The template provides beautiful, syntax-highlighted code boxes that integrate with the design system.

\subsection{Python Examples}

\subsubsection{Matrix Operations}

\begin{pythoncode}
import numpy as np

# Create a matrix
A = np.array([[1, 2], [3, 4]])
B = np.array([[5, 6], [7, 8]])

# Matrix multiplication
C = np.dot(A, B)
print(f"Result:\n{C}")

# Eigenvalues
eigenvalues = np.linalg.eigvals(A)
print(f"Eigenvalues: {eigenvalues}")
\end{pythoncode}

\subsubsection{Data Analysis}

\begin{codebox}[language=Python]{Statistical Analysis}
import pandas as pd
import matplotlib.pyplot as plt

# Load data
df = pd.read_csv('data.csv')

# Calculate statistics
mean = df['values'].mean()
std = df['values'].std()

# Plot histogram
plt.hist(df['values'], bins=30)
plt.title(f'Distribution (mean={mean:.2f}, std={std:.2f})')
plt.show()
\end{codebox}

\subsection{C++ Examples}

\subsubsection{Template Function}

\begin{cppcode}
// Template function for finding maximum value
#include <vector>
#include <algorithm>
#include <iostream>

template<typename T>
T findMax(const std::vector<T>& vec) {
    if (vec.empty()) {
        throw std::invalid_argument("Empty vector");
    }
    return *std::max_element(vec.begin(), vec.end());
}

int main() {
    std::vector<int> numbers = {3, 1, 4, 1, 5, 9};
    int max_val = findMax(numbers);
    std::cout << "Maximum: " << max_val << std::endl;
    return 0;
}
\end{cppcode}

\subsection{Java Examples}

\begin{javacode}
public class BinarySearch {
    public static int search(int[] arr, int target) {
        int left = 0, right = arr.length - 1;
        
        while (left <= right) {
            int mid = left + (right - left) / 2;
            
            if (arr[mid] == target) {
                return mid;  // Found
            } else if (arr[mid] < target) {
                left = mid + 1;
            } else {
                right = mid - 1;
            }
        }
        return -1;  // Not found
    }
}
\end{javacode}

\subsection{Inline Code Blocks}

For short snippets without title or line numbers:

\begin{codeblock}[language=bash]
# Install dependencies
pip install numpy scipy matplotlib

# Run the script
python analysis.py --input data.csv
\end{codeblock}

\subsection{MATLAB Example}

\begin{codebox}[language=Matlab]{Signal Processing}
% Generate signal
fs = 1000;              % Sampling frequency
t = 0:1/fs:1-1/fs;      % Time vector
f = 5;                  % Frequency
signal = sin(2*pi*f*t);

% Apply FFT
Y = fft(signal);
P2 = abs(Y/length(signal));

% Plot
plot(t, signal);
xlabel('Time (s)');
ylabel('Amplitude');
title('Sine Wave Signal');
\end{codebox}

\subsection{SQL Example}

\begin{codebox}[language=SQL]{Database Query}
-- Select students with high grades
SELECT 
    s.student_id,
    s.name,
    AVG(g.grade) AS avg_grade
FROM students s
JOIN grades g ON s.student_id = g.student_id
WHERE g.semester = 'Fall 2024'
GROUP BY s.student_id, s.name
HAVING AVG(g.grade) >= 90
ORDER BY avg_grade DESC;
\end{codebox}

\subsection{Custom Language Example}

You can specify any language supported by \texttt{listings}:

\begin{codebox}[language=R]{Statistical Modeling in R}
# Load data
data <- read.csv("experiment.csv")

# Fit linear model
model <- lm(response ~ predictor1 + predictor2, data = data)

# Summary statistics
summary(model)

# Plot residuals
plot(model$residuals)
abline(h = 0, col = "red", lty = 2)
\end{codebox}
