% theorems.tex
% Theorem-like environments using tcolorbox for better visual presentation
% NOTE: Colors are defined in config/theme.tex

%%%%%%%%%%%%%%%%%%%%%%%%%%%%%%%%%%%%%%%%%%%%%%%%%%%%%%%%%%%%%%%%%%%%%%%
% COLORED THEOREM ENVIRONMENTS (using tcolorbox)
%%%%%%%%%%%%%%%%%%%%%%%%%%%%%%%%%%%%%%%%%%%%%%%%%%%%%%%%%%%%%%%%%%%%%%%

% Theorem (royal blue) - formal mathematical statements
\newtcbtheorem[number within=section]{theorem}{Theorem}{
  colback=theorem-blue!5,
  colframe=theorem-blue,
  coltitle=white,
  fonttitle=\bfseries,
  boxrule=0.8pt,
  arc=3pt,
  left=8pt,
  right=8pt,
  top=8pt,
  bottom=8pt
}{thm}

% Definition (teal) - foundational concepts
\newtcbtheorem[use counter from=theorem]{definition}{Definition}{
  colback=definition-teal!5,
  colframe=definition-teal,
  coltitle=white,
  fonttitle=\bfseries,
  boxrule=0.8pt,
  arc=3pt,
  left=8pt,
  right=8pt,
  top=8pt,
  bottom=8pt
}{def}

% Lemma (blue variant) - supporting theorems
\newtcbtheorem[use counter from=theorem]{lemma}{Lemma}{
  colback=theorem-blue!5,
  colframe=theorem-blue!85,
  coltitle=white,
  fonttitle=\bfseries,
  boxrule=0.8pt,
  arc=3pt,
  left=8pt,
  right=8pt,
  top=8pt,
  bottom=8pt
}{lem}

% Proposition (blue variant) - formal statements
\newtcbtheorem[use counter from=theorem]{proposition}{Proposition}{
  colback=theorem-blue!5,
  colframe=theorem-blue!85,
  coltitle=white,
  fonttitle=\bfseries,
  boxrule=0.8pt,
  arc=3pt,
  left=8pt,
  right=8pt,
  top=8pt,
  bottom=8pt
}{prop}

% Corollary (blue variant) - consequences of theorems
\newtcbtheorem[use counter from=theorem]{corollary}{Corollary}{
  colback=theorem-blue!5,
  colframe=theorem-blue!85,
  coltitle=white,
  fonttitle=\bfseries,
  boxrule=0.8pt,
  arc=3pt,
  left=8pt,
  right=8pt,
  top=8pt,
  bottom=8pt
}{cor}

% Example (slate gray) - illustrations
\newtcbtheorem[use counter from=theorem]{example}{Example}{
  colback=example-slate!5,
  colframe=example-slate,
  coltitle=white,
  fonttitle=\bfseries,
  boxrule=0.8pt,
  arc=3pt,
  left=8pt,
  right=8pt,
  top=8pt,
  bottom=8pt
}{ex}

% Remark (purple) - observations and notes
\newtcbtheorem[use counter from=theorem]{remark}{Remark}{
  colback=remark-purple!5,
  colframe=remark-purple,
  coltitle=white,
  fonttitle=\bfseries,
  boxrule=0.8pt,
  arc=3pt,
  left=8pt,
  right=8pt,
  top=8pt,
  bottom=8pt
}{rem}

% Note (purple variant) - additional observations
\newtcbtheorem[use counter from=theorem]{note}{Note}{
  colback=remark-purple!5,
  colframe=remark-purple!85,
  coltitle=white,
  fonttitle=\bfseries,
  boxrule=0.8pt,
  arc=3pt,
  left=8pt,
  right=8pt,
  top=8pt,
  bottom=8pt
}{note}

%%%%%%%%%%%%%%%%%%%%%%%%%%%%%%%%%%%%%%%%%%%%%%%%%%%%%%%%%%%%%%%%%%%%%%%
% SIMPLE ENVIRONMENTS (for backward compatibility with amsthm syntax)
%%%%%%%%%%%%%%%%%%%%%%%%%%%%%%%%%%%%%%%%%%%%%%%%%%%%%%%%%%%%%%%%%%%%%%%

% These use the standard amsthm syntax: \begin{exampleplain}[Optional Title]
% Restore simple theorem environments for those that don't need fancy boxes
\theoremstyle{definition}
\newtheorem{exampleplain}{Example}[section]
\newtheorem{remarkplain}[exampleplain]{Remark}
\newtheorem{definitionplain}[exampleplain]{Definition}

% Make standard environments use the plain versions by default for backward compatibility
% Users can use the tcolorbox versions by explicitly calling them with {title}{label}
\let\example\exampleplain
\let\endexample\endexampleplain
\let\remark\remarkplain
\let\endremark\endremarkplain
\let\definition\definitionplain
\let\enddefinition\enddefinitionplain

%%%%%%%%%%%%%%%%%%%%%%%%%%%%%%%%%%%%%%%%%%%%%%%%%%%%%%%%%%%%%%%%%%%%%%%
% UTILITY BOXES (non-numbered environments)
%%%%%%%%%%%%%%%%%%%%%%%%%%%%%%%%%%%%%%%%%%%%%%%%%%%%%%%%%%%%%%%%%%%%%%%

% Warning box (red) - critical alerts
\newtcolorbox{warningbox}{
  colback=warning-red!5,
  colframe=warning-red,
  coltitle=white,
  title=Warning,
  fonttitle=\bfseries,
  boxrule=0.8pt,
  arc=3pt,
  left=8pt,
  right=8pt,
  top=8pt,
  bottom=8pt
}

% Important box (orange) - key information
\newtcolorbox{importantbox}{
  colback=important-orange!5,
  colframe=important-orange,
  coltitle=white,
  title=Important,
  fonttitle=\bfseries,
  boxrule=0.8pt,
  arc=3pt,
  left=8pt,
  right=8pt,
  top=8pt,
  bottom=8pt
}

% Question box (amber) - inquiry and problems
\newtcolorbox{questionbox}{
  colback=question-amber!5,
  colframe=question-amber,
  coltitle=white,
  title=Question,
  fonttitle=\bfseries,
  boxrule=0.8pt,
  arc=3pt,
  left=8pt,
  right=8pt,
  top=8pt,
  bottom=8pt
}

% Answer box (green) - for homework/exercise solutions
\newtcolorbox{answerbox}{
  colback=success-green!5,
  colframe=success-green,
  coltitle=white,
  title=Answer,
  fonttitle=\bfseries,
  boxrule=0.8pt,
  arc=3pt,
  left=8pt,
  right=8pt,
  top=8pt,
  bottom=8pt
}

% Solution box (green variant) - detailed solutions
\newtcolorbox{solutionbox}{
  colback=success-green!5,
  colframe=success-green!85,
  coltitle=white,
  title=Solution,
  fonttitle=\bfseries,
  boxrule=0.8pt,
  arc=3pt,
  left=8pt,
  right=8pt,
  top=8pt,
  bottom=8pt
}

% Tip box (teal) - helpful hints
\newtcolorbox{tipbox}{
  colback=secondary-main!5,
  colframe=secondary-main,
  coltitle=white,
  title=Tip,
  fonttitle=\bfseries,
  boxrule=0.8pt,
  arc=3pt,
  left=8pt,
  right=8pt,
  top=8pt,
  bottom=8pt
}

%%%%%%%%%%%%%%%%%%%%%%%%%%%%%%%%%%%%%%%%%%%%%%%%%%%%%%%%%%%%%%%%%%%%%%%
% CODE BOXES (with syntax highlighting)
%%%%%%%%%%%%%%%%%%%%%%%%%%%%%%%%%%%%%%%%%%%%%%%%%%%%%%%%%%%%%%%%%%%%%%%

% Load tcolorbox library for listings integration
\tcbuselibrary{listings,skins,breakable}

% Code box with title (using listings)
\newtcblisting{codebox}[2][]{
  colback=bg-light,
  colframe=example-slate,
  coltitle=white,
  title=#2,
  fonttitle=\bfseries,
  boxrule=0.8pt,
  arc=3pt,
  left=0pt,
  right=8pt,
  top=8pt,
  bottom=8pt,
  breakable,
  skin=enhanced jigsaw,
  listing only,
  listing options={
    basicstyle=\ttfamily\small\color{text-primary},
    keywordstyle=\color{primary-main}\bfseries,
    commentstyle=\color{text-secondary}\itshape,
    stringstyle=\color{success-green},
    numberstyle=\tiny\color{text-secondary},
    numbers=left,
    numbersep=8pt,
    xleftmargin=20pt,
    tabsize=4,
    breaklines=true,
    showstringspaces=false,
    #1
  }
}

% Inline code box (no title, no line numbers)
\newtcblisting{codeblock}[1][]{
  colback=bg-light,
  colframe=divider,
  boxrule=0.5pt,
  arc=2pt,
  left=8pt,
  right=8pt,
  top=8pt,
  bottom=8pt,
  breakable,
  skin=enhanced jigsaw,
  listing only,
  listing options={
    basicstyle=\ttfamily\small\color{text-primary},
    keywordstyle=\color{primary-main}\bfseries,
    commentstyle=\color{text-secondary}\itshape,
    stringstyle=\color{success-green},
    numbers=none,
    tabsize=4,
    breaklines=true,
    showstringspaces=false,
    #1
  }
}

% Language-specific code boxes (convenient shortcuts)
\newtcblisting{pythoncode}[1][]{
  listing options={language=Python},
  colback=bg-light,
  colframe=example-slate,
  coltitle=white,
  title=Python Code,
  fonttitle=\bfseries,
  boxrule=0.8pt,
  arc=3pt,
  left=0pt,
  right=8pt,
  top=8pt,
  bottom=8pt,
  breakable,
  skin=enhanced jigsaw,
  listing only,
  listing options={
    language=Python,
    basicstyle=\ttfamily\small\color{text-primary},
    keywordstyle=\color{primary-main}\bfseries,
    commentstyle=\color{text-secondary}\itshape,
    stringstyle=\color{success-green},
    numberstyle=\tiny\color{text-secondary},
    numbers=left,
    numbersep=8pt,
    xleftmargin=20pt,
    tabsize=4,
    breaklines=true,
    showstringspaces=false,
    #1
  }
}

\newtcblisting{javacode}[1][]{
  listing options={language=Java},
  colback=bg-light,
  colframe=example-slate,
  coltitle=white,
  title=Java Code,
  fonttitle=\bfseries,
  boxrule=0.8pt,
  arc=3pt,
  left=0pt,
  right=8pt,
  top=8pt,
  bottom=8pt,
  breakable,
  skin=enhanced jigsaw,
  listing only,
  listing options={
    language=Java,
    basicstyle=\ttfamily\small\color{text-primary},
    keywordstyle=\color{primary-main}\bfseries,
    commentstyle=\color{text-secondary}\itshape,
    stringstyle=\color{success-green},
    numberstyle=\tiny\color{text-secondary},
    numbers=left,
    numbersep=8pt,
    xleftmargin=20pt,
    tabsize=4,
    breaklines=true,
    showstringspaces=false,
    #1
  }
}

\newtcblisting{cppcode}[1][]{
  listing options={language=C++},
  colback=bg-light,
  colframe=example-slate,
  coltitle=white,
  title=C++ Code,
  fonttitle=\bfseries,
  boxrule=0.8pt,
  arc=3pt,
  left=0pt,
  right=8pt,
  top=8pt,
  bottom=8pt,
  breakable,
  skin=enhanced jigsaw,
  listing only,
  listing options={
    language=C++,
    basicstyle=\ttfamily\small\color{text-primary},
    keywordstyle=\color{primary-main}\bfseries,
    commentstyle=\color{text-secondary}\itshape,
    stringstyle=\color{success-green},
    numberstyle=\tiny\color{text-secondary},
    numbers=left,
    numbersep=8pt,
    xleftmargin=20pt,
    tabsize=4,
    breaklines=true,
    showstringspaces=false,
    #1
  }
}

%%%%%%%%%%%%%%%%%%%%%%%%%%%%%%%%%%%%%%%%%%%%%%%%%%%%%%%%%%%%%%%%%%%%%%%
% STANDARD THEOREM ENVIRONMENTS (commented out, kept for reference)
%%%%%%%%%%%%%%%%%%%%%%%%%%%%%%%%%%%%%%%%%%%%%%%%%%%%%%%%%%%%%%%%%%%%%%%
%%%%%%%%%%%%%%%%%%%%%%%%%%%%%%%%%%%%%%%%%%%%%%%%%%%%%%%%%%%%%%%%%%%%%%%

% % Standard amsthm theorem environments (plain style)
% \newtheorem{theorem}{Theorem}[section]
% \newtheorem{lemma}[theorem]{Lemma}
% \newtheorem{proposition}[theorem]{Proposition}
% \newtheorem{corollary}[theorem]{Corollary}
% 
% % Definition environments
% \newtheorem{definition}[theorem]{Definition}
% \newtheorem{example}[theorem]{Example}
% \newtheorem{remark}[theorem]{Remark}
% \newtheorem{note}[theorem]{Note}

% Proof environment (already defined by amsthm, but can be customized)
% \renewcommand{\qedsymbol}{$\blacksquare$}
