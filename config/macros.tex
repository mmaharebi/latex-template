% macros.tex
% Custom mathematical notation macros for scalars, vectors, matrices, and tensors
% NOTE: Derivative operators, quantum mechanics, and some utilities now use the
%       production-ready 'physics' package for better robustness and maintainability

%%%%%%%%%%%%%%%%%%%%%%%%%%%%%%%%%%%%%%%%%%%%%%%%%%%%%%%%%%%%%%%%%%%%%%%
% PHYSICAL-STYLE NOTATION MACROS (Custom - Unique Underline Style)
%%%%%%%%%%%%%%%%%%%%%%%%%%%%%%%%%%%%%%%%%%%%%%%%%%%%%%%%%%%%%%%%%%%%%%%

% Scalar: italic (default math mode, explicit wrapper for consistency)
\newcommand{\scalar}[1]{\mathit{#1}}

% Advanced underline helpers for proper spacing and kerning
\makeatletter
% Double overline helper (not used but included for completeness)
\newcommand{\dbloverline}[1]{\overline{\dbl@overline{#1}}}
\newcommand{\dbl@overline}[1]{\mathpalette\dbl@@overline{#1}}
\newcommand{\dbl@@overline}[2]{%
  \begingroup
  \sbox\z@{$\m@th#1\overline{#2}$}%
  \ht\z@=\dimexpr\ht\z@-2\dbl@adjust{#1}\relax
  \box\z@
  \ifx#1\scriptstyle\kern-\scriptspace\else
  \ifx#1\scriptscriptstyle\kern-\scriptspace\fi\fi
  \endgroup
}

% Double underline helper with proper spacing
\newcommand{\dblunderline}[1]{\@@underline{\dbl@underline{#1}}}
\newcommand{\dbl@underline}[1]{\mathpalette\dbl@@underline{#1}}
\newcommand{\dbl@@underline}[2]{%
  \begingroup
  \sbox\z@{$\m@th#1\@@underline{#2}$}%
  \dp\z@=\dimexpr\dp\z@-2\dbl@adjust{#1}\relax
  \box\z@
  \ifx#1\scriptstyle\kern-\scriptspace\else
  \ifx#1\scriptscriptstyle\kern-\scriptspace\fi\fi
  \endgroup
}

% Adjustment factor for different math styles
\newcommand{\dbl@adjust}[1]{%
  \fontdimen8
  \ifx#1\displaystyle\textfont\else
  \ifx#1\textstyle\textfont\else
  \ifx#1\scriptstyle\scriptfont\else
  \scriptscriptfont\fi\fi\fi 3
}

% Kerning correction helper
\newcommand{\mathcorr}[1]{\mathpalette\math@corr{#1}}
\newcommand{\math@corr}[2]{%
  \begingroup
  \sbox\z@{$\m@th#1#2$}\sbox2{$\m@th#1#2_{}\kern-\scriptspace$}%
  \kern\dimexpr\wd\z@-\wd\tw@\relax
  \endgroup
}
\makeatother

% Vector: bold + single underline with proper spacing and kerning
\renewcommand{\vec}[1]{{\bm{%
  \mspace{0.5mu}%
  \underline{\mspace{-0.5mu}#1_{}\kern-\scriptspace\mspace{-0.5mu}}%
  \mspace{0.5mu}%
  \mathcorr{#1}%
}}}

% Matrix: bold + double underline with proper spacing and kerning
\newcommand{\mat}[1]{{\bm{%
  \mspace{0.5mu}%
  \dblunderline{\mspace{-0.5mu}#1_{}\kern-\scriptspace\mspace{-0.5mu}}%
  \mspace{0.5mu}%
  \mathcorr{#1}%
}}}

% Tensor (order >= 3): bold calligraphic (optional)
\newcommand{\tensor}[1]{\boldsymbol{\mathcal{#1}}}

%%%%%%%%%%%%%%%%%%%%%%%%%%%%%%%%%%%%%%%%%%%%%%%%%%%%%%%%%%%%%%%%%%%%%%%
% CONVENIENCE MACROS
%%%%%%%%%%%%%%%%%%%%%%%%%%%%%%%%%%%%%%%%%%%%%%%%%%%%%%%%%%%%%%%%%%%%%%%

% Differential (physics package provides \dd, but we keep \dif for compatibility)
\newcommand{\dif}{\mathrm{d}}
\newcommand{\e}{\mathrm{e}}
\newcommand{\imag}{\mathrm{j}}  % Imaginary unit (engineering convention)

% Probability and Statistics
\newcommand{\Prob}[1]{\mathrm{P}\left(#1\right)}                    % Probability: P(A)
\newcommand{\given}{\,|\,}                                          % Conditional: P(A|B)
\newcommand{\Expect}[1]{\mathrm{E}\left[#1\right]}                  % Expectation: E[X]
\newcommand{\Var}[1]{\mathrm{Var}\left(#1\right)}                   % Variance: Var(X)
\newcommand{\Cov}[2]{\mathrm{Cov}\left(#1, #2\right)}               % Covariance: Cov(X,Y)
\newcommand{\Corr}[2]{\mathrm{Corr}\left(#1, #2\right)}             % Correlation: Corr(X,Y)
\newcommand{\StdDev}[1]{\mathrm{SD}\left(#1\right)}                 % Standard deviation: SD(X)
\newcommand{\bias}[1]{\mathrm{bias}\left(#1\right)}                 % Bias: bias(θ̂)
\newcommand{\MSE}[1]{\mathrm{MSE}\left(#1\right)}                   % Mean squared error: MSE(θ̂)

% Distributions
\newcommand{\Normal}[2]{\mathcal{N}\left(#1, #2\right)}             % Normal distribution: N(μ,σ²)
\newcommand{\Uniform}[2]{\mathcal{U}\left(#1, #2\right)}            % Uniform distribution: U(a,b)
\newcommand{\Bernoulli}[1]{\mathrm{Bernoulli}\left(#1\right)}       % Bernoulli(p)
\newcommand{\Binomial}[2]{\mathrm{Binomial}\left(#1, #2\right)}     % Binomial(n,p)
\newcommand{\Poisson}[1]{\mathrm{Poisson}\left(#1\right)}           % Poisson(λ)
\newcommand{\Exponential}[1]{\mathrm{Exp}\left(#1\right)}           % Exponential(λ)

% Convergence
\newcommand{\convdist}{\xrightarrow{d}}                             % Convergence in distribution
\newcommand{\convprob}{\xrightarrow{p}}                             % Convergence in probability
\newcommand{\convas}{\xrightarrow{a.s.}}                            % Almost sure convergence
\newcommand{\convLp}[1]{\xrightarrow{L^{#1}}}                       % L^p convergence

% Independence and identically distributed
\newcommand{\iid}{\stackrel{\text{iid}}{\sim}}                      % i.i.d. notation
\newcommand{\ind}{\perp\!\!\!\perp}                                 % Independence symbol

% Indicator function (using dsfont package)
\newcommand{\Ind}[1]{\mathds{1}_{#1}}                               % Indicator: 𝟙_A
\newcommand{\One}{\mathds{1}}                                       % Indicator: 𝟙

% Optimization
\DeclareMathOperator*{\argmin}{arg\,min}                            % Argmin: argmin
\DeclareMathOperator*{\argmax}{arg\,max}                            % Argmax: argmax
\newcommand{\minimize}{\operatorname{minimize}}                      % minimize
\newcommand{\maximize}{\operatorname{maximize}}                      % maximize
\newcommand{\subjectto}{\operatorname{subject\,to}}                 % subject to

% Set Theory
\newcommand{\set}[1]{\left\{#1\right\}}                            % Set notation: {·}
\newcommand{\setbuilder}[2]{\left\{#1 \mid #2\right\}}             % Set builder: {x | condition}
\newcommand{\card}[1]{\left|#1\right|}                              % Cardinality: |A|
\newcommand{\powerset}[1]{\mathcal{P}\left(#1\right)}               % Power set: 𝒫(A)
\newcommand{\union}{\cup}                                            % Union: ∪
\newcommand{\Union}{\bigcup}                                         % Big union: ⋃
\newcommand{\intersect}{\cap}                                        % Intersection: ∩
\newcommand{\Intersect}{\bigcap}                                     % Big intersection: ⋂
\newcommand{\comp}[1]{#1^{\complement}}                              % Complement: Aᶜ
% Note: \setminus already defined in LaTeX for set difference: \

% Complexity and Asymptotic Notation
\newcommand{\Oh}{\mathcal{O}}                                        % Big-O: 𝒪
\newcommand{\oh}{o}                                                  % Little-o: o
% Note: \Omega, \omega, \Theta already defined as Greek letters
% Use them directly for complexity notation

% Linear Algebra Operators
% Note: \rank already defined by physics package
\DeclareMathOperator{\nullity}{nullity}                             % Nullity: nullity(A)
\DeclareMathOperator{\Span}{span}                                   % Span: span(v₁,...,vₙ)
\DeclareMathOperator{\Ker}{ker}                                     % Kernel: ker(T)
\DeclareMathOperator{\Image}{im}                                    % Image: im(T)
\DeclareMathOperator{\Null}{null}                                   % Null space: null(A)
\DeclareMathOperator{\Row}{row}                                     % Row space: row(A)
\DeclareMathOperator{\Col}{col}                                     % Column space: col(A)
\DeclareMathOperator{\diag}{diag}                                   % Diagonal: diag(a₁,...,aₙ)
\newcommand{\transpose}[1]{#1^{\top}}                                % Transpose: Aᵀ
\newcommand{\hermitian}[1]{#1^{\dagger}}                             % Hermitian conjugate: A†
\newcommand{\inv}[1]{#1^{-1}}                                        % Inverse: A⁻¹
% Note: \det already defined in LaTeX, use it directly

% Note: \abs{} and \norm{} are now provided by physics package
% Note: \tr (trace) and \rank are provided by physics package
% If you need custom sizing behavior, uncomment these:
% \renewcommand{\abs}[1]{\left|#1\right|}
% \renewcommand{\norm}[1]{\left\|#1\right\|}

%%%%%%%%%%%%%%%%%%%%%%%%%%%%%%%%%%%%%%%%%%%%%%%%%%%%%%%%%%%%%%%%%%%%%%%
% DERIVATIVES (Using physics package)
%%%%%%%%%%%%%%%%%%%%%%%%%%%%%%%%%%%%%%%%%%%%%%%%%%%%%%%%%%%%%%%%%%%%%%%

% The physics package provides:
% - \dv{f}{x}       : df/dx (total derivative)
% - \dv[n]{f}{x}    : d^n f/dx^n (nth derivative)
% - \pdv{f}{x}      : ∂f/∂x (partial derivative)
% - \pdv[n]{f}{x}   : ∂^n f/∂x^n (nth partial)
% - \pdv{f}{x}{y}   : ∂²f/∂x∂y (mixed partial)

% Partial time derivative shorthand
\newcommand{\partialt}{\partial_t}

%%%%%%%%%%%%%%%%%%%%%%%%%%%%%%%%%%%%%%%%%%%%%%%%%%%%%%%%%%%%%%%%%%%%%%%
% SYMBOLIC OPERATORS
%%%%%%%%%%%%%%%%%%%%%%%%%%%%%%%%%%%%%%%%%%%%%%%%%%%%%%%%%%%%%%%%%%%%%%%

% Definition symbol
\newcommand{\eqdef}{\stackrel{\text{def}}{=}}

%%%%%%%%%%%%%%%%%%%%%%%%%%%%%%%%%%%%%%%%%%%%%%%%%%%%%%%%%%%%%%%%%%%%%%%
% COMMON VECTORS AND MATRICES
%%%%%%%%%%%%%%%%%%%%%%%%%%%%%%%%%%%%%%%%%%%%%%%%%%%%%%%%%%%%%%%%%%%%%%%

% Common vectors
\newcommand{\vzero}{\vec{0}}
\newcommand{\vi}{\vec{i}}
\newcommand{\vj}{\vec{j}}
\newcommand{\vk}{\vec{k}}

% Common matrices
\newcommand{\mzero}{\mat{0}}
\newcommand{\midentity}{\mat{I}}

%%%%%%%%%%%%%%%%%%%%%%%%%%%%%%%%%%%%%%%%%%%%%%%%%%%%%%%%%%%%%%%%%%%%%%%
% UNIT VECTORS
%%%%%%%%%%%%%%%%%%%%%%%%%%%%%%%%%%%%%%%%%%%%%%%%%%%%%%%%%%%%%%%%%%%%%%%

% Unit vector notation
\newcommand{\uvec}[1]{\hat{\vec{#1}}}
\newcommand{\uevec}[1]{{\vec{e}}_{#1}}  % Alternative shorthand

% Unit vectors in Cartesian coordinates
\newcommand{\xhat}{\hat{\vec{x}}}
\newcommand{\yhat}{\hat{\vec{y}}}
\newcommand{\zhat}{\hat{\vec{z}}}

% Unit vectors in cylindrical coordinates
\newcommand{\rhohat}{\hat{\vec{\rho}}}
\newcommand{\phihat}{\hat{\vec{\phi}}}

% Unit vectors in spherical coordinates
\newcommand{\Rhat}{\hat{\vec{R}}}
\newcommand{\thetahat}{\hat{\vec{\theta}}}

%%%%%%%%%%%%%%%%%%%%%%%%%%%%%%%%%%%%%%%%%%%%%%%%%%%%%%%%%%%%%%%%%%%%%%%
% VECTOR OPERATIONS
%%%%%%%%%%%%%%%%%%%%%%%%%%%%%%%%%%%%%%%%%%%%%%%%%%%%%%%%%%%%%%%%%%%%%%%

% Note: physics package defines \vdot as derivative, so we use \vecdot for vector dot product
\newcommand{\vecdot}{\bullet}
\newcommand{\vproj}[2]{\text{proj}_{\vec{#1}}\vec{#2}}
\newcommand{\vang}[2]{\angle(\vec{#1},\vec{#2})}

%%%%%%%%%%%%%%%%%%%%%%%%%%%%%%%%%%%%%%%%%%%%%%%%%%%%%%%%%%%%%%%%%%%%%%%
% VECTOR CALCULUS (Now using physics package!)
%%%%%%%%%%%%%%%%%%%%%%%%%%%%%%%%%%%%%%%%%%%%%%%%%%%%%%%%%%%%%%%%%%%%%%%

% The physics package provides:
% - \grad    : ∇ (gradient)
% - \div     : ∇· (divergence)
% - \curl    : ∇× (curl)
% - \laplacian : ∇² (Laplacian)

% Note: physics already defines \vnabla, we can just use it directly

% Legacy aliases (optional - uncomment if you prefer old names):
% \renewcommand{\Div}{\div}

%%%%%%%%%%%%%%%%%%%%%%%%%%%%%%%%%%%%%%%%%%%%%%%%%%%%%%%%%%%%%%%%%%%%%%%
% MATRIX OPERATORS
%%%%%%%%%%%%%%%%%%%%%%%%%%%%%%%%%%%%%%%%%%%%%%%%%%%%%%%%%%%%%%%%%%%%%%%

% Note: physics package provides \tr for trace and \rank for rank
% No need to redefine them

%%%%%%%%%%%%%%%%%%%%%%%%%%%%%%%%%%%%%%%%%%%%%%%%%%%%%%%%%%%%%%%%%%%%%%%
% QUANTUM MECHANICS NOTATION (Now using physics package!)
%%%%%%%%%%%%%%%%%%%%%%%%%%%%%%%%%%%%%%%%%%%%%%%%%%%%%%%%%%%%%%%%%%%%%%%

% The physics package provides robust bra-ket notation:
% - \ket{a}          : |a⟩
% - \bra{b}          : ⟨b|
% - \braket{a}{b}    : ⟨a|b⟩
% - \ketbra{a}{b}    : |a⟩⟨b|
% - \expval{A}       : ⟨A⟩ (expectation value)
% - \expval{A}{\psi} : ⟨ψ|A|ψ⟩
% - \mel{a}{B}{b}    : ⟨a|B|b⟩ (matrix element)
% - \comm{A}{B}      : [A,B] (commutator)
% - \acomm{A}{B}     : {A,B} (anticommutator)

% Legacy alias for expectation (optional):
% \newcommand{\expectation}[1]{\expval{#1}}

% Operator hat notation
\newcommand{\hop}[1]{\hat{#1}}

%%%%%%%%%%%%%%%%%%%%%%%%%%%%%%%%%%%%%%%%%%%%%%%%%%%%%%%%%%%%%%%%%%%%%%%
% COMPLEX ANALYSIS NOTATION
%%%%%%%%%%%%%%%%%%%%%%%%%%%%%%%%%%%%%%%%%%%%%%%%%%%%%%%%%%%%%%%%%%%%%%%

% Real and imaginary parts (Gothic/Fraktur font)
% Redefine the existing \Re and \Im commands to use Fraktur
\renewcommand{\Re}{\mathfrak{Re}}
\renewcommand{\Im}{\mathfrak{Im}}

% Common sets
\newcommand{\R}{\mathbb{R}}
\newcommand{\C}{\mathbb{C}}
\newcommand{\N}{\mathbb{N}}
\newcommand{\Z}{\mathbb{Z}}

%%%%%%%%%%%%%%%%%%%%%%%%%%%%%%%%%%%%%%%%%%%%%%%%%%%%%%%%%%%%%%%%%%%%%%%
% LEGACY/DEPRECATED COMMANDS (Commented out - now using physics package)
%%%%%%%%%%%%%%%%%%%%%%%%%%%%%%%%%%%%%%%%%%%%%%%%%%%%%%%%%%%%%%%%%%%%%%%

% These commands are now provided by the physics package.
% Uncomment if you need backward compatibility with old documents:

% % Differential operators (replaced by physics \dv, \pdv)
% \newcommand{\diff}{\mathrm{d}}
% \newcommand{\Diff}{\mathrm{D}}
% \newcommand{\pderiv}[2]{\frac{\partial #1}{\partial #2}}

% % Absolute value and norm (replaced by physics \abs, \norm)
% % Note: physics package provides automatic sizing
% \renewcommand{\abs}[1]{\left|#1\right|}
% \renewcommand{\norm}[1]{\left\|#1\right\|}

% % Bra-ket notation (replaced by physics package)
% \renewcommand{\ket}[1]{\left| #1 \right\rangle}
% \renewcommand{\bra}[1]{\left\langle #1 \right|}
% \renewcommand{\braket}[2]{\left\langle #1 \middle| #2 \right\rangle}
% \renewcommand{\ketbra}[2]{\left| #1 \right\rangle\left\langle #2 \right|}
% \renewcommand{\expectation}[1]{\left\langle #1 \right\rangle}

% % Commutators (replaced by physics \comm, \acomm)
% \renewcommand{\comm}[2]{\left[\hop{#1}, \hop{#2}\right]}
% \renewcommand{\acomm}[2]{\left\{\hop{#1}, \hop{#2}\right\}}

% % Trace operator (replaced by physics \tr)
% \DeclareMathOperator{\trace}{tr}
